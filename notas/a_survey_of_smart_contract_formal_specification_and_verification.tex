Los llamados {\it smart contracts} son programas de computadora
que permiten a los usuarios automatizar las acciones de las 
{\it BlockChains}. Actualmente los {\it smart contracts}
han cobrado relevancía en multiples sectores, como lo son
la cadena de suministros, finanzas, servicios legales médicos
y {\bf electorales}. La importancía de estos está sujeta a
una verificación.

%% Hablar de que son las especificaciones formales y pq no hay
%% En las BC

BlockChains como herramienta que surgió del deseo de
almacenar información con marca de tiempo en
una cadena criptograficamente segura.

BlockChain: Registro distribuido compartido entre
los nodos en una red {\it peer-to-peer} que sigue
algún {\it protocolo de consenso}.

Ethereum comenzó la llamada BlockChain 2.0
que expande las capacidades y aplicaciones
al introducir{\it smart contracts} que cumplen
con ser {\it Turing Completos}.

{\it Smart Contract} acuñado por Nich Szabo.
[198] Nick Szabo. 1997. Formalizing and Securing
Relationships on Public Networks. First Monday 2, 9 (Sep. 1997).

Ej. En Ethereum, cada {\it contrato} esta asociado
con una {\it dirección}, un {\it balance} y
codigo ejecutable.  Las funciones son llamadas
a ejecución a través de mensajes de transacción enviados
por usuarios de la BC. Ethereum usa Solidity como lenguaje de SC.



Understanding the DAO attack. paper
Parity Multsig Wallet
King of the Ether throne 


En este árticulo se centran en las distintas
técnicas que se han seguido para verificar
SC. Con especial atención a propiedades
de dominio especifico ({\it domain-specific properties})
y demás formalizmos que cubren dichas especificaciones
para así verificarlas.


Formal Verification: Una especificación es un conjunto de propiedades
que describen el comportamiento deseado de los SC,
usualmente siguiendo las intenciones de los desarrolladores. 

Models and specification:

EL PAPER SOLO HABLA DE UN VERIFICACIÖN DE SC, no se mete con:
Scalability, Consensus, interaction with IoT, etc

Establecer los nexos entre los distintos
tipos de propiedades y los formalismos
que los soportan.

% Sección II
Smart contract modeling formalism
Hay dos categorías, caja negra vs caja blanca:

Cajas blancas son sistemas deductvos, nos permiten ver paso a paso TLA+
\begin{enumerate}
\item A nível de contrato

  Se suelen enfocar en el comportamiento de alto nível
  de los SC. No suelen considerar detalles técnicos
  de su implementación y ejecución.

  Describen la interacción entre SC y agentes externos
  que por lo general son usuarios de la BC.

  Los detalles del SC son abstraidos en funciones publicas
  que son invocadas por el usuario, cuya ejecución tiene
  resultados observables como lo son: eventos emitidos, o
  un cambio en el {\it estado} de la BC.

\item A nivel de programa

  Analísis en la implementación del contrato (código fuente
  o código compilado). Por lo tanto hay tantos como plataformas
  de SC.  Este enfoque permiten razonar sobre detalles de bajo
  nivel de la ejecución de los SC. 
\end{enumerate}

%% Sección 3 Smart Contract Specifications
Se entiende como especificaciónes de SC
como propiedades escritas en lenguaje formal
sobre el comportamiento de los SC.

Semejanzas entre las especificaciones
{\it model-oriented} y {\it property-oriented}
utilizadas en las lógicas de Hoare y
{\it contract-level} y {\it program-level}
para hablar de las especificaciones de los SC.
Siendo de alto y bajo nivel respectivamente.

Checar bien la sección 3.1.1 pag 12. Se hace mención
de que lógicas ayudan para formalizar el comportamiento
de BC.

Contract level:

Ambas propiedades, {\it Safety} y {\it liveness}, estan
relacionadas con la idea de tiempo. Como los
SC actúan a través de este.
\begin{itemize}
\item {\it Safety}: ``Nunca ocurre nada malo''. Expresan
invarianza. Estan relacionados los {\it deadlock-freedom}.
Definen valores permisibles para las variables de estado,
que establecen invariantes sobre el balance del SC:
especificando control de acceso a ciertas funciones o
definiendo un posible orden de eventos.
\item {\it Liveness}: ``Algo bueno eventualmente pasara'',
posibilidad que un evento progrese. Que en el contexto
de los SC por lo general se refiera a que
ls BC sea capaz de dar cryptocurrency. Hay varias y
estan asociadas al algoritmo de consenso: 
\begin{itemize}
\item {\it Liquidity}
\item posibilidad de llegar a otro estado.
\end{itemize}

%%Linear temporal Logic

%% Computational Tree Logic,
%Alternanting Temporal, Stategic Logics.
% Dynamic Logic, Linead Dynamic logic,

%% -----
% Superioridad de Denotic and Defeasible Logic.
% Pues permite verse como una especificación
% ejecutable y, por lo tanto, facilita el monitoreo.
%% ----
%Implementación en COQ, checar 185

Program level:
Rescatan las logicas de Hoare,
expresan corrección parcial.

%Checar las out-of-gas exception

Definir, Precondiciones y Postcondiciones,
algo natural que sucede en {\it Solidity}.

Smart contract verification te permite ver si el contraro
es vulnerable, Gas-inefficient o un honeypot.

%% Checar effectively callback free

%% Clasificación de propiedades por dominios.
\begin{itemize}
\item seguridad
\begin{itemize}
\item Liquidity. Si falla se llaman ``greedy'',
      los fondos no pueden ser liberados bajo ninfuna condición.
\item Atomicity
\item single-entrancy
\item acces control
\item arithmetic correctness
\item reasonable resource consumption
%% Propone que se checando si las ejecuciones son
%% linerizability y serializability
\end{itemize}

%%
\item Privacidad. Se suelen establecer
protocolos criptograficos, como esquemas commitment y
de timed commitment. La correctud de
un commitment SC es descrito en terminos
del conocimiento de los secretos(contraseñas
que tienen tiempo para responder) de cada uno
y los cambios en sus balances. 
\item Finanzas. 
\item Social games
\item Asset Tracking 
\end{itemize}
\end{itemize}


%% 4. Smart Contract FOrmal Verification Techniques
Técnicas, herramientas y frameworks utilizados
para hacer verificación de smart contracts categorizadas en:

\begin{enumerate}
\item Model Checking

Su popularidad radica en que hay que
modelar y especificar formalmente los SC
y ya hay frameworks que trabajan con esos conceptos.

Se pueden tener diferentes caracteristicas, al meter logica temporal, como lo son:
concurrency, no determinismo, limitaciones del tiemp.

Existen limitaciones expresivas del lenguaje de los model checkers.
\item Theorem Proving

Consiste en codificar el sistema y sus propiedades a una
lógica matemática. Y es cuando un {\it theorem prover}
intenta dar una derivación formal(prueba) que satisfaga las propiedades
basadas en axiomas y reglas de inferencia del LF. A diferencía de los
MC que solo verifican sistemas de estados finitos, los TP verifica
sistemas infinitos.

Se necesita interacción humana y expertis. Se utiliza Coq, Isabbelle/GOL y Agda.

el caso de Concert
[25] A. Arusoaie. 2020. Certifying Findel derivatives for blockchain. arXiv:2005.13602

%% CASo de Nielsen y Spitters voting contract invariant
% embed code in Coq en Scilla. Paper 186 Safer Smart Contract
% programming with Schilla. 2019
\item Symbolic and Concolic Execution   
\item Program Verification
\item Run Time Verification and Testing

%% ¿Cuáles son las Diferencias entre SC y Legal Contracts?
\end{enumerate}

Por lo general temporal properties son verificadas
por un model checker. Patrones de vulnerabilidad
en CFG o SC
son identificados usando ejecución simbolica
Theorem Proving es usado para dar una especificación
precisa y verificar {\it correctness} de los SC
(usando tripletas de Hoare), lenguajes de programación
y frameworkds. 

%% 5. Discussión
Checar 5.1 para más resultados

%%Global clock

%% 6. Conclusión 

%% Gas consumption
%% Temporal properties:
   % liveness
