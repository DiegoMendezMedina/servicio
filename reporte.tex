\documentclass[14pt,letterpaper]{article}
\usepackage[margin=2cm,includefoot]{geometry}
\usepackage[spanish]{babel}

\usepackage[utf8]{inputenc}

%\usepackage{mathptmx}

\usepackage[colorlinks = true,
            linkcolor = blue,
            urlcolor  = blue,
            citecolor = blue,
            anchorcolor = blue]{hyperref}


\usepackage{fancyhdr}
\pagestyle{fancy}

\usepackage{xcolor}
\usepackage{amssymb}
\usepackage{hyperref}
\usepackage{amsmath}
\usepackage{listings}
\usepackage{graphicx}
\graphicspath{{./images/}}

\author{}
\date{}
\title{Aspectos teóricos de las blockchain en sistemas de votación
\thanks{\small{Material desarrollado bajo el proyecto UNAM-2024.}}}
\lhead{Aspectos teóricos de las blockchain en sistemas de votación}
\rhead{Volver al \hyperlink{toc}{índice}}
\begin{document}

\ttfamily
\maketitle
\rmfamily

%\section{Resumen}

\addtocontents{toc}{\protect\hypertarget{toc}{}}
\tableofcontents
\newpage

\section{Introducción}

\section{Sistemas blockchain}
\subsection{Que son}
\noindent
Los sistemas de blockchains o tecnologías de blockchain
son el uso conjunto/simultáneo de: 1. la estructura de datos
de cadena de bloques({\it blockchain}), 2. algún algoritmo distribuido de consenso,
3. criptografía de llave pública y, recientemente, 4. contratos
inteligentes ({\it smart contracts}); en una red {\it peer-to-peer}.
Todos reunidos forman una red anoníma.

\footnotesize
{\bf (*Nota) pendiente a desarrollar: }
Motivaciones economicas desde su fundación. Moneda digital y descentralizada.
\normalsize

%% Estructura de datos
Sabemos que para $n$ agentes en una red distribuida el sistema puede tolerar hasta
$\frac{n}{4}$ agentes maliciosos, dicho protocolo asume que se conocen
todos los agentes. La {\it blockchain} como estructura de datos es creada con el fin 
de resolver el problema del consenso Bizantino tolerante a fallos
para un número arbitrario de agentes en una red anónima. 

El primer acercamiento al desarrollo de estas fue la creación de las
marcas de tiempo ({\it timestamps}) que denoten el momento en el que
algúna operación fue efectuada. La respuesta se encontró en la
criptografía. {\small {\bf (*Nota) pendiente a desarrollar:} Que es un Hash}

Como ya mencionamos la motivación
inicial de los sistemas blockchain era crear una moneda digital.
Así, llamamos bloque a la información de las operaciones/transacciones
con la supuesta moneda. Para poder preservar o simular el tiempo de forma efectiva
falta alimentar el hash con otra cosa. Cada
marca de tiempo debe probar que la información de cada bloque existió y de algúna forma
estar vinculada con el pasado. Salvo por el bloque inicial, único bloque cuya marca
de tiempo es el hash con la información inicial, el resto de bloques creará
su marca de tiempo al hacer un hash sobre la información del bloque y
la marca de tiempo anterior. Al hash ser único y seguiendo este procedimiento
para cada bloque/tiempo nuevo es que tenemos la estructura de datos blockchain.

{\small {\bf (*Nota) pendiente a desarrollar:} Mencionar que se pueden
  tener varias ramas pero por que la más larga es la confiable.
  Se pueden usar las blockchains sin tener sistemas distribuidos pero
  como ya vimos no es el motivo de su creación. }

%% p2p
{\small {\bf (*Nota) pendiente a desarrollar:} Redes p2p con la
  estructura de datos y criptografía de llave pública nos permiten transmitir
  información cifrada y anónima para cada nodo. Ventajas sobre
  redes cliente-servidor: Cualquier agente puede desconectarse de
  la red, es descentralizado, encaminar la discusión al problema del consenso}

%% Proof of work, etc
{\small {\bf (*Nota) pendiente a desarrollar:} Introducir el rol de verificadores:
  encargados de crear nuevos bloques, y
  el llamado {\it zero block} con sus funcionalidades: primeros verificadores,
  configuraciones de red, primeras monedas. Ver que existen diferentes
  algoritmos de consenso en las BC: {\bf PoW, PoS}, PoT}

%% Smart contracts
{\small {\bf (*Nota) pendiente a desarrollar:} {\it SmartContract: Programas
    que ejecutan los nodos de los sistemas de blockchain. Permiten automatizar
    procesos de las blockchains}. Hablar de sus diferencias con el paso de mensajes y 
  por que Bitcoin funcionaba sin los contratos. Por que son utiles y hablar
  de Solidity. 
  }

%% BlockChains existentes

Las notas de la pendiente sección se siguen de \cite{bitcoin} y \cite{ethereum}.

\subsection{Sus vulnerabilidades}

\noindent
En los sistemas de blockchain existen dos tipos de vulnerabilidades.
El primero esta asociado con el protocolo mismo. En este tipo
se incluyen: uso de funciones criptográficas mal implementadas y
formas de atacar los protocolos de consenso para controlar la red.
Por otro lado el segundo surge de añadir smart contracts al sistema. Al
ser Touring Completos crean problemas que dependen del
protocolo de consenso pero también en las billeteras digitales y sus constantes
interacciones. 

{\small {\bf (*Nota) pendiente a desarrollar:} Hablar de las vulnerabilidades
  a las que estan expuestos los algoritmos de protocolos y escenarios donde
  se pierde el control de la blockchain.}

{\small {\bf (*Nota) pendiente a desarrollar:} Hablar 1) del caso DAO, como es que
  se podía ejecutar código inseguro/no verificado, y 2) de como las primitivas
  de la EVM pueden crear bugs que un adversario puede utilizar.}

Para ambas ideas pendientes de esta sección se utilizará. \cite{scANDformalMethods},
\cite{surveysc}

\section{Métodos formales en los sistemas de blockchain}

%% A nivel de protocolo 
\subsection{Formalizando consenso}

{\small {\bf (*Nota) pendiente a desarrollar:} Hablaremos de las ventajas
  ecólogicas de PoS sobre PoW, de como por la naturaleza de PoS el adversario
  se compromete financiaremente, y sobre la primera formalización de PoS en coq}
\cite{PoSCoq}.

%% A nivel de smart contract
\subsection{Creando mejores contratos inteligentes}

{\small {\bf (*Nota) pendiente a desarrollar:} Clasificación que la
  academía le ha dado a los problemas, antes mencionados, en los smart contracts.
  1) Creación de lenguajes más seguros, 2) Analísis de programa (low level, seguridad),
  3) Criptografía y 4) Nuevos protocolos } \cite{scANDformalMethods}\cite{surveysc}.
           
\section{Voto electrónico}

\subsection{Antecedentes}

{\small {\bf (*Nota) pendiente a desarrollar:} Motivaciones de voto electronico.
Casos donde se ha votado electronicamente (casos cliente-servidor y supuestamente con
blockchain), hablar brevemente de los sistemas donde se realizaron. Suiza chvote mixnets.
Virginia y Russia con blockchain.}

\subsection{Votando con blockhains}

{\small {\bf (*Nota) pendiente a desarrollar:} Hablar del esquema de votos, la
  llamada independencia de software y como las blockchains cubren de
  mejor forma el e-vote } \cite{votingScheme}\cite{BCEV}\cite{ETHVote}

{\small {\bf (*Nota) pendiente a desarrollar:} Traer a discusión lo antes explicado
con el articulo del MIT \cite{MIT}.}
           
\subsection{Métodos Formales en la Votación Electronica}

{\small {\bf (*Nota) pendiente a desarrollar:} Los dos casos exitosos y completos
  que existen sobre verificación en herramientas de votación han sido en sistemas
  cliente-servidor. Hablaremos de ellos dos y de que manera podríamos llevarlos a
  la blockchain y como cubre problemas planteados por el MIT}\cite{homotallying}
\cite{mixNets}\cite{MIT}.

\subsubsection{Homomorphic}
\subsubsection{MixNets}

\section{Caso especifico}

{\small {\bf (*Nota) pendiente a desarrollar:}
  Una vez vista la importancía de los smartcontracts y se sus vulnerabilidades en el
  sistema más popular; veremos el caso de Rholang, programa concurrente cuyo fundamento
  es el cálculo $\rho$, una extensión del cálculo $\pi$. El lenguaje inicialmente
  creo su sistema de blockchain, llamado RChain, pero también puede ser
  utilizado en otros sistemas. Se introducira el cálculo rho, veremos las primitivas
  del lenguaje/mostrar ejercicios que puede resolver (los del tutorial más unos propios)
  y se mostrara como se puede llegar a un consenso PoS (el consenso verificable en coq)}
\cite{rhoCalc}\cite{rhoLang}

\subsection{El cálculo $\rho$}
\subsection{Rholang}
\subsection{PoS en Rholang}

\section{Conclusión}

\section{Bibliografía}

\bibliographystyle{plain} 
\bibliography{./doc/refs.bib} 

\end{document}
